% pdflatex -shell-escape nyst2d && ac nyst2d.pdf &
\documentclass[t]{beamer}
\usepackage[utf8]{inputenc}

\usetheme{CCM}
\usepackage{epstopdf}  % handle EPS
\usepackage{bm}

% for MATLAB code display...
%\usepackage{mcode}   % matlabcentral, no nice colors (?) but can embed latex
% see: https://tex.stackexchange.com/questions/180222/how-to-change-font-size-for-specific-lstlisting
% OR...
\usepackage{xcolor}
\usepackage{minted}            % pygmentize code listings; needs -shell-escape
\definecolor{codebg}{rgb}{0.95,0.95,0.95}
\setminted{fontsize=\tiny,bgcolor=codebg}
\newminted[matc]{matlab}{}    % here matc is the macro name, matlab the language
\newmintinline[mati]{matlab}{}


% alex macros
\input{../rgb}
\newcommand{\ft}[1]{\frametitle{#1}}
\newcommand{\bi}{\begin{itemize}}
\newcommand{\ei}{\end{itemize}}
\newcommand{\ben}{\begin{enumerate}}
\newcommand{\een}{\end{enumerate}}
\newcommand{\be}{\begin{equation}}
\newcommand{\ee}{\end{equation}}
\newcommand{\ba}{\begin{align}}
\newcommand{\ea}{\end{align}}
\newcommand{\bc}{\begin{center}}
\newcommand{\ec}{\end{center}}
\newcommand{\mbf}[1]{{\bm #1}}           % requires bm package
\newtheorem{thm}{Theorem}
\newcommand{\ig}[2]{\includegraphics[#1]{#2}}
\newcommand{\tbox}[1]{{\mbox{\tiny #1}}}
\newcommand{\who}[1]{{\scriptsize \textcolor{darkgreen}{(#1)}}}  % cite
\newcommand{\com}[1]{{\scriptsize \textcolor{purple}{#1}}}      % comment
\newcommand{\co}[1]{{\scriptsize \tt #1}}          % code
\newcommand{\vg}{\vspace{2ex}}
\newcommand{\sg}{\vspace{1ex}}
\newcommand{\hg}{\vspace{0.5ex}}
\newcommand{\dr}[1]{{\color{darkred}#1}}    % for bullets
\newcommand{\rb}{\ensuremath{\textcolor{darkred}{\bullet\;}}\ }
\newcommand{\bmp}[1]{\begin{minipage}{#1}}
\newcommand{\bmpt}[1]{\begin{minipage}[t]{#1}}
\newcommand{\emp}{\end{minipage}}
\newcommand{\pig}[2]{\bmp{#1}\includegraphics[width=#1]{#2}\emp} % mp-fig, nogap
\newcommand{\pigm}[3]{\bmp{#1}\href{#3}{\includegraphics[width=#1]{#2}}\emp} % w/ movie
\newcommand{\ora}[1]{{\color{orange} #1}}
\newcommand{\gre}[1]{{\color{green} #1}}
\newcommand{\yel}[1]{{\color{yellow} #1}}
\newcommand{\sr}[1]{{\scriptsize #1}}
\newcommand{\eps}{\epsilon}
\newcommand{\qqquad}{\qquad\qquad}
\newcommand{\qqqquad}{\qqquad\qqquad}
\newcommand{\R}{\mathbb{R}}
\newcommand{\Z}{\mathbb{Z}}
\newcommand{\N}{\mathbb{N}}
\newcommand{\emach}{\epsilon_\tbox{mach}}
\DeclareMathOperator{\im}{Im}
\DeclareMathOperator{\re}{Re}
\newcommand{\bigO}{{\cal O}}
\newcommand{\pO}{{\partial\Omega}}
\newcommand{\xx}{\mbf{x}}
\newcommand{\yy}{\mbf{y}}
\newcommand{\uu}{\mbf{u}}
\newcommand{\nn}{\mbf{n}}
\newcommand{\Srep}{{\cal S}}
\newcommand{\Drep}{{\cal D}}

%%%%%%%%%%%%%%%%%%%%%%%%%%%%%%%%%%%%%%%%%%%%%%%%%%%%%%%%%%%


% TO DO:

% beamerCCMtheme.sty : add noframe fragile option







%%%%%%%%%%%%%%%%%%%%%%%%%%%%%%%%%%%%%%%%%%%%%%%%%%%%%%%%%%%%%%%%
\title{2D boundary integral equations and the Nystr\"om method}
\date{Computational Tools 2024 BIE workshop. Day 1, 6/10/24}
\author{\textbf{Alex Barnett}\inst{1}
and \textbf{Fruzsina Agocs}\inst{1}}
\institute{\inst{1} Center for Computational Mathematics, Flatiron Institute, Simons Foundation}

\begin{document}

\begin{frame}
	\titlepage
\end{frame}

\begin{frame}\ft{Integral equations on 1D interval}

  Given: i) function $\sigma(t)$ defined on interval $[0,2\pi)$,
\hfill\com{periodic: $\sigma(2\pi)=\sigma(0)$, etc}
    
\qquad\quad ii) ``kernel'' function $k(t,s)$ defined on square $[0,2\pi)^2$,

\sg
  
Integral {\em operator} $K$ acts on $\sigma$ to give another function $K\sigma$:

\hg

\bmp{3in}
\quad $(K\sigma)(t) := \int_0^{2\pi} k(t,s) \sigma(s) ds, \quad t \in [0,2\pi)$
\emp
\hfill
\bmp{1.4in}
\com{continuous analog of}

\vspace{-0.7ex}

\com{matrix-vector prod.\ $A\xx$}
\emp

\sg

\bmp{3in}
Integral {\em equation}: \quad $(I+K)\sigma =f$, ie
$$\sigma(t) + \int_0^{2\pi} k(t,s) \sigma(s) ds = f(t),
\quad t \in [0,2\pi)$$
  
  \emp
  \bmp{1.7in}
  \ig{width=1.7in}{skie}

\com{analog of lin.\ sys.\
  $A\xx=\mbf{b}$}
  \emp
  
  %write out:
  \sg

\quad \com{Fredholm ``second kind'' (due to presence of $I$, otherwise called ``first kind'')}

\sg

If kernel %$k(\cdot,\cdot)$
continuous, or ``weakly'' singular (integrable), $K$ is {\em compact}:

\rb
eigenvalues ($K\phi_k=\lambda_k\phi_k$) discrete, with $\lim_{k\to\infty}\lambda_k = 0$
%accumulate only at $0$

\quad \com{unless some $\lambda_k = -1$,
  %uniqueness:
  2nd kind IE has at most one soln: Nul $(I+K) = \{0\}$}
%  nullspace trivial

\rb Nul $(I+K) = \{0\}$
%has only trivial soln
\;$\Rightarrow$\;
{\em existence} of solution for {\em any} $f$ \hfill\com{Fredholm Alternative}

\quad \com{like square matrix (finite-dim), recall:\; uniqueness $\Rightarrow$
  consistent for any RHS}

Contrast 1st kind IE $K\sigma = f$ is ill-posed problem (unstable)!

%\rb $K$ is the norm limit of finite-rank operators \com{basically matrices}

\com{See references for lots of beautiful theory, precise statements}
%  see books for precise definitions}


%\com{called ``Fredholm'' (if $k$ integrable, eg only weakly singular),

  
%\rb treat domain as periodic: $2\pi$ glued back to $0$, so $f(2\pi) = f(0)$, etc

%Soon $t$ will {\em parameterize} a closed boundary curve  \com{eg $\xx(t) = (\cos t, \sin t)$}

\end{frame}


\begin{frame}\ft{Nystr\"om discretization of 2nd kind IE on interval}

  Simplest quadrature for %$2\pi$-
  periodic function: periodic trapezoid rule (PTR)
  
$\int_0^{2\pi} f(t) dt
\approx
\sum_{j=1}^N \frac{2\pi}{N} f\bigl(\frac{2\pi j}{N} \bigr)
=
\sum_{j=1}^N w_j f(t_j)
$
\com{$w_j$=weights, \, $t_j$=nodes}

\com{For $f$ smooth, superalgebraically convergent: error $=\bigO(N^{-p})$ for all $p$}

\sg

Apply quad to integral in 2nd kind IE:
\hfill\com{call the resulting approx soln $\tilde\sigma$}

$\tilde\sigma(t) + \sum_{j=1}^N k(t,t_j) w_j \tilde\sigma(t_j)  = f(t),
\quad t \in [0,2\pi) \qquad (*)$

\sg
  
Holds for all $t$; in particular, holds at each $t_i$, $i=1,\dots,N$, giving:

$\sigma_i + \sum_{j=1}^N k(t_i,t_j)w_j \sigma_j = f(t_i),
\quad i=1,\dots,N$
\hfill\com{where $\sigma_j:=\tilde\sigma(t_j)$}

\sg

Write as: \qquad $A\bm{\sigma} = \bm{f}$
\hfill\com{$N\times N$ lin sys, entries $a_{ij} = \delta_{ij} + k(t_i,t_j)w_j$, and
  $f_j:=f(t_j)$}

\sg

Sometimes in BIE, node values $\{\sigma_j\}_{j=1}^N$ enough.

If not, interpolate from them to any $t\in[0,2\pi)$. Two approaches:

  \quad \com{either: rearrange (*) to give $\tilde\sigma(t) = \dots$, the
    Nystr\"om interpolant (rarely used)}
  
  \quad \com{or (more common): use high-order local Lagrange
    (or global spectral) interpolation}

\sg
  
  Thm \who{Anselone, Kress,\dots}: error of node values (and Nystr\"om
  interpolant) same order as that of quadrature rule applied
  to integrand $k(t,\cdot)\sigma$.
% $\{\sigma_j\}_{j=1}^N$ 
  
\end{frame}



\begin{frame}[fragile] \ft{Demo Nystr\"om}  % need fragile for any verbatim env
\vspace{-3ex}
  
  \hfill\com{see {\tt day1/nyst\_demos.m}}   % verb failed even w/ protect :(

% \begin{lstlisting}   % could add code comments with latex, eg:  % §\mcommentfont $k(t_i,t_j)w_j$ for $i,j=1,\dots,N$§
%\end{lstlisting}
  % Or use minted...
  \begin{matc}
kfun = @(s,t) exp(3*cos(t-s));                 % smooth convolutional kernel, periodic domain [0,2pi)
ffun = @(t) cos(5*t+1);                        % smooth data (RHS) func
N = 30;                                        % number of unknowns: should study convergence as N grows...
tj = 2*pi/N*(1:N); wj = 2*pi/N*ones(1,N);      % PTR nodes and weights, row vecs
A = eye(N) + bsxfun(kfun,tj',tj)*diag(wj);     % identity plus fill k(t_i,t_j)w_j for i,j=1..N
rhs = ffun(tj');                               % col vec
sigmaj = A\rhs;                                % dense direct square solve (pivoted LU), gives col vec
\end{matc}
  
  \pig{3.5in}{code/nyst_conv}
  \bmp{1.0in}
  \com{$f(t)$ and $k(t,s)$ smooth}

  \com{so $\sigma(t)$ smooth}

\com{PTR spectral conv for smooth}  
  
  \com{$\Rightarrow$ get spectral conv}
\emp
  
\pause

% attempt to label graphs
%\hspace{1.8in} \mati|ffun = @(t) abs(sin(t))| \hfill \mati|kfun = @(s,t) 10*sin(abs(t-s)/2).^3|

\bmp{2.3in}
\rb But $f$ and/or $k$ nonsmooth?

worse (merely algebraic)
convergence using plain PTR rule:
\emp
\hfill\pig{2.3in}{code/nyst_nonsm}

\rb 

\end{frame}



\begin{frame}\ft{PDEs and fundamental solution}
\end{frame}

\begin{frame}\ft{Potential theory}
\end{frame}

\begin{frame}\ft{}
\end{frame}

\begin{frame}\ft{}
\end{frame}

\begin{frame}\ft{}
\end{frame}

\begin{frame}\ft{}
\end{frame}


\begin{frame}\ft{Recap}

\rb
  
  \rb Nystr\"om discretization gets $\sigma(t_j)$ \com{interpolate from them to other $t$}


  
\end{frame}


\begin{frame}\ft{Resources}  % RRRRRRRRRRRRRRRRRRRRRRRRRRRRRRRRRRRRRRRRRRRRRRR

Numerical analysis (more mathematical flavor):

\hg

{\em Linear Integral Equations}, R. Kress, (1999, Springer). Ch. 6 \& 12.

\hg

{\em The Numerical Solution of Integral Equations of the Second Kind},
K. E. Atkinson, (1997, CUP).

\sg

Practical side:

\hg

``High-order accurate methods for Nystr\"om discretization of integral equations on smooth curves in the plane'', S Hao, AH Barnett, PG Martinsson, P Young.
{\em Adv. Comput. Math.} {\bf 40}, 245--272 (2014).

\hg

\url{https://users.flatironinstitute.org/~ahb/BIE/}

\hg

\url{https://github.com/ahbarnett/BIEbook} \hfill \com{not much there yet\dots}

\end{frame}


\end{document}
