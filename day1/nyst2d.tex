% pdflatex -shell-escape nyst2d && ac nyst2d.pdf &
\documentclass[t]{beamer}
\usepackage[utf8]{inputenc}

\usepackage{../templates/beamerthemeCCM}
\usepackage{epstopdf}  % handle EPS
\usepackage{bm}
\usepackage{wasysym}   % emojis
\usepackage{tikz}          % try big arrow
\usetikzlibrary{arrows.meta, shapes.arrows, positioning}  % https://tex.stackexchange.com/questions/555562/how-to-draw-a-large-arrow-with-text-between-two-tikz-pictures


% for MATLAB code display...
%\usepackage{mcode}   % matlabcentral, no nice colors (?) but can embed latex
% see: https://tex.stackexchange.com/questions/180222/how-to-change-font-size-for-specific-lstlisting
% OR...
\usepackage{xcolor}
\usepackage{minted}            % pygmentize code listings; needs -shell-escape
\usepackage{ragged2e}
\usepackage{tabularx}
\usepackage{mathtools} % under- and overbrace
\usepackage[makeroom]{cancel} % For cancelling out terms in math
\definecolor{codebg}{rgb}{0.95,0.95,0.95}
\setminted{fontsize=\tiny,bgcolor=codebg}
\newminted[matc]{matlab}{}    % here matc is the macro name, matlab the language
\newmintinline[mati]{matlab}{}

% alex macros
\input{../templates/rgb}
\newcommand{\ft}[1]{\frametitle{#1}}
\newcommand{\bi}{\begin{itemize}}
\newcommand{\ei}{\end{itemize}}
\newcommand{\ben}{\begin{enumerate}}
\newcommand{\een}{\end{enumerate}}
\newcommand{\be}{\begin{equation}}
\newcommand{\ee}{\end{equation}}
\newcommand{\ba}{\begin{align}}    % seems to fail in beamer
\newcommand{\ea}{\end{align}}
\newcommand{\bea}{\begin{eqnarray}}
\newcommand{\eea}{\end{eqnarray}}
\newcommand{\bc}{\begin{center}}
\newcommand{\ec}{\end{center}}
\newcommand{\mbf}[1]{{\bm #1}}           % requires bm package
\newtheorem{thm}{Theorem}
\newcommand{\ig}[2]{\includegraphics[#1]{#2}}
\newcommand{\tbox}[1]{{\mbox{\tiny #1}}}
\newcommand{\who}[1]{{\scriptsize \textcolor{darkgreen}{(#1)}}}  % cite
\newcommand{\com}[1]{{\scriptsize \textcolor{purple}{#1}}}      % comment
\newcommand{\co}[1]{{\scriptsize \tt #1}}          % code
\newcommand{\vg}{\vspace{2ex}}
\newcommand{\sg}{\vspace{1ex}}
\newcommand{\hg}{\vspace{0.5ex}}
\newcommand{\hng}{\vspace{-0.5ex}}
\newcommand{\dr}[1]{{\color{darkred}#1}}    % for bullets
\newcommand{\rb}{\ensuremath{\textcolor{darkred}{\bullet\;}}\ }
\newcommand{\bmp}[1]{\begin{minipage}{#1}}
\newcommand{\bmpt}[1]{\begin{minipage}[t]{#1}}
\newcommand{\emp}{\end{minipage}}
\newcommand{\sfrac}[2]{\mbox{\small $\frac{#1}{#2}$}}
\newcommand{\pig}[2]{\bmp{#1}\includegraphics[width=#1]{#2}\emp} % mp-fig, nogap
\newcommand{\pigm}[3]{\bmp{#1}\href{#3}{\includegraphics[width=#1]{#2}}\emp} % w/ movie
\newcommand{\ora}[1]{{\color{CCMorange} #1}}
\newcommand{\gre}[1]{{\color{green} #1}}
\newcommand{\yel}[1]{{\color{yellow} #1}}
\newcommand{\sr}[1]{{\scriptsize #1}}
\newcommand{\eps}{\epsilon}
\newcommand{\qqquad}{\qquad\qquad}
\newcommand{\qqqquad}{\qqquad\qqquad}
\newcommand{\R}{\mathbb{R}}
\newcommand{\Z}{\mathbb{Z}}
\newcommand{\N}{\mathbb{N}}
\newcommand{\emach}{\epsilon_\tbox{mach}}
\DeclareMathOperator{\im}{Im}
\DeclareMathOperator{\re}{Re}
\newcommand{\bigO}{{\cal O}}
\newcommand{\pO}{{\partial\Omega}}
\newcommand{\xx}{\mbf{x}}
\newcommand{\yy}{\mbf{y}}
\newcommand{\uu}{\mbf{u}}
\newcommand{\nn}{\mbf{n}}
\newcommand{\Srep}{{\cal S}}
\newcommand{\Drep}{{\cal D}}

%%%%%%%%%%%%%%%%%%%%%%%%%%%%%%%%%%%%%%%%%%%%%%%%%%%%%%%%%%%


% TO DO:

% beamerCCMtheme.sty : add noframe fragile option ? can't get working


%%%%%%%%%%%%%%%%%%%%%%%%%%%%%%%%%%%%%%%%%%%%%%%%%%%%%%%%%%%%%%%%
\title{2D boundary integral equations and the Nystr\"om method}
\date{Computational Tools 2024 BIE workshop. Day 1, 6/10/24}
\author{\textbf{Alex Barnett}\inst{1}
and \textbf{Fruzsina Agocs}\inst{1}}
\institute{\inst{1} Center for Computational Mathematics, Flatiron Institute, Simons Foundation}

\begin{document}

\begin{frame}
	\titlepage
\end{frame}

\begin{noframe}\ft{Integral equations on 1D interval}

  Given: i) function $\sigma(t)$ defined on interval $[0,2\pi)$,
\hfill\com{periodic: $\sigma(2\pi)=\sigma(0)$, etc}
    
\qquad\quad ii) ``kernel'' function $k(t,s)$ defined on square $[0,2\pi)^2$,

\sg
  
Integral {\em operator} $K$ acts on $\sigma$ to give another function $K\sigma$:

\sg

\bmp{3in}
\quad $(K\sigma)(t) := \int_0^{2\pi} k(t,s) \sigma(s) ds, \quad t \in [0,2\pi)$
\emp
\hfill
\bmp{1.4in}
\com{continuous analog of}

\vspace{-0.7ex}

\com{matrix-vector prod.\ $A\xx$}
\emp

\sg
\pause

\bmp{3in}
Integral {\em equation}: \quad $(I+K)\sigma =f$, ie
$$\sigma(t) + \int_0^{2\pi}\! k(t,s) \sigma(s) ds = f(t),
\;\; t \in [0,2\pi)$$
  
  \emp
  \bmp{1.7in}
  \ig{width=1.7in}{skie}

\com{analog of lin.\ sys.\
  $A\xx=\mbf{b}$}
  \emp
  
  %write out:
  \sg

\quad \com{Fredholm ``second kind'' (due to presence of $I$, otherwise called ``first kind'')}

\sg
\pause

If kernel %$k(\cdot,\cdot)$
continuous, or ``weakly'' singular (integrable), $K$ is {\bf compact}:

\sg

\rb
eigenvalues ($K\phi_k=\lambda_k\phi_k$) discrete, with $\lim_{k\to\infty}\lambda_k = 0$
%accumulate only at $0$

\quad \com{unless some $\lambda_k = -1$,
  %uniqueness:
  2nd kind IE has at most one soln: Nul $(I+K) = \{0\}$}
%  nullspace trivial

\rb Nul $(I+K) = \{0\}$
%has only trivial soln
\;$\Rightarrow$\;
{\em existence} of solution for {\em any} $f$ \hfill\com{Fredholm Alternative}

\quad \com{like square matrix (finite-dim), recall:\; uniqueness $\Rightarrow$
  consistent for any RHS}

\pause

Contrast 1st kind IE $K\sigma = f$ is ill-posed problem (unstable)!

%\rb $K$ is the norm limit of finite-rank operators \com{basically matrices}

%\com{See references for lots of beautiful theory, precise statements}
%  see books for precise definitions}


%\com{called ``Fredholm'' (if $k$ integrable, eg only weakly singular),

  
%\rb treat domain as periodic: $2\pi$ glued back to $0$, so $f(2\pi) = f(0)$, etc

%Soon $t$ will {\em parameterize} a closed boundary curve  \com{eg $\xx(t) = (\cos t, \sin t)$}

\end{noframe}


\begin{noframe}\ft{Nystr\"om discretization of 2nd kind IE on interval}

  Simplest quadrature for %$2\pi$-
  periodic funcs: periodic trapezoid rule (PTR)

\hg
 \quad  
$\int_0^{2\pi} f(t) dt
\approx
\sum_{j=1}^N \frac{2\pi}{N} f\bigl(\frac{2\pi j}{N} \bigr)
=
\sum_{j=1}^N w_j f(t_j)
$
\hfill \com{$w_j$=weights, \, $t_j$=nodes}

\hg

\com{For $f$ smooth, superalgebraically
  convergent
 (``spectral''): error $=\bigO(N^{-p})$ for any $p$}

\sg
\pause

Apply quadr.\ to integral in 2nd kind IE:
\hfill\com{call the resulting approx soln $\tilde\sigma$}

\sg

\quad $\tilde\sigma(t) + \sum_{j=1}^N k(t,t_j) w_j \tilde\sigma(t_j)  = f(t),
\quad t \in [0,2\pi) \qquad (*)$

\sg
\pause

Holds for all $t$; in particular, holds at each $t_i$, $i=1,\dots,N$, giving:

\sg

\quad $\sigma_i + \sum_{j=1}^N k(t_i,t_j)w_j \sigma_j = f(t_i),
\quad i=1,\dots,N$
\hfill\com{where $\sigma_j:=\tilde\sigma(t_j)$}

\sg
\pause

Write as: \qquad $A\bm{\sigma} = \bm{f}$
\hfill\com{$N\times N$ lin sys, entries $a_{ij} = \delta_{ij} + k(t_i,t_j)w_j$, and
  $f_j:=f(t_j)$}

\hg

\quad\com{solve? dense direct $\bigO(N^3)$;
  dense iter.\ $\bigO(N^2)$; fast iter.\ $\approx \bigO(N)$; fast direct
  $\approx \bigO(N^{(d+1)/2})$}

\quad\com{Why want 2nd kind? eigs$(A)$ accumulate only at $+1$ $\Rightarrow$ iterative converges fast}

\sg
\pause

Sometimes for BIE (eg, far-field eval), node values $\{\sigma_j\}_{j=1}^N$ suffice.

If not, interpolate from them to any $t\in[0,2\pi)$. Two approaches:

  \bmp{4in}
\rb \com{either: rearrange (*) to give $\tilde\sigma(t) = \dots$, called
    ``Nystr\"om interpolant'' (rare)}
  
\rb \com{or (common): use local high-order Lagrange
    (or global spectral) interpolation:}
\emp
\pig{.8in}{flocpoly}

\end{noframe}



\begin{frame}[fragile] \ft{Demo Nystr\"om on interval (1D)}  % need fragile for any verbatim env
\vspace{-5ex}
  
  \hfill\com{{\tt day1/code/nyst1d\_demos.m}}   % verb failed even w/ protect :(

% \begin{lstlisting}   % could add code comments with latex, eg:  % §\mcommentfont $k(t_i,t_j)w_j$ for $i,j=1,\dots,N$§
%\end{lstlisting}
  % Or use minted...
  \begin{matc}
kfun = @(t,s) exp(3*cos(t-s));                 % smooth convolutional kernel, periodic domain [0,2pi)
ffun = @(t) cos(5*t+1);                        % smooth data (RHS) func
N = 30;                                        % number of unknowns: should study convergence as N grows...
t = 2*pi/N*(1:N); w = 2*pi/N*ones(1,N);        % PTR nodes and weights, row vecs
A = eye(N) + bsxfun(kfun,t',t)*diag(w);        % identity plus fill k(t_i,t_j)w_j for i,j=1..N
rhs = ffun(t');                                % col vec
sigmaj = A\rhs;                                % dense direct square solve (pivoted LU), gives col vec
\end{matc}
  
  \pig{3.5in}{nyst_conv}
  \hfill
\bmp{1.1in}
\com{``self-convergence'':}

\hng

\com{use $N{=}40$ as ``true''}

\sg

\com{$f$ and $k$ smooth}

\hng

\com{$\Rightarrow\;$ $\sigma$ smooth}

\hng

\com{$\Rightarrow\;$ spectral conv?}
\emp

{\bf Thm.} \who{Anselone, Kress,\dots}: error at node values (and Nystr\"om
interpolant) same order as that of quadrature rule applied
to integrand $k(t,\cdot)\sigma$.
% $\{\sigma_j\}_{j=1}^N$ 
%  \quad\com{Eg: $k$ and $f$ smooth $\;\Rightarrow\;$  $\sigma$ smooth
%    $\;\Rightarrow\;$ PTR Nystr\"om spectral conv.}
  


\pause
\sg

% attempt to label graphs
%\hspace{1.8in} \mati|ffun = @(t) abs(sin(t))| \hfill \mati|kfun = @(s,t) 10*sin(abs(t-s)/2).^3|

\bmp{2.3in}
\rb Then, $f$ or $k$ nonsmooth?

worse (here {\em algebraic})
convergence using plain PTR rule:

\sg

\com{Qu: why does order appear to improve at end?}

\emp
\hfill\pig{2.3in}{nyst_nonsm}

\end{frame}



\begin{noframe}\ft{Laplace fundamental solution in $\R^2$}

Eg PDE: Poisson eqn $\Delta u = g$
\hfill\com{$\Delta:=(\partial/\partial x_1)^2+(\partial/\partial x_2)^2$ Laplacian}

\quad\com{notation: $\xx := (x_1,x_2) \in \R^2$ is a point. This frees up $\yy \in\R^2$ as another point (not y-coord!)}

\quad\com{not well-posed unless add BC! \quad BIEs are good for {\em homogeneous} PDEs (driving $g\equiv 0$)}

\pause

\bmp{1.5in}
A well-posed$^*$ BVP:

\sg

\quad\com{$^*$exists, unique,}

\hng

\quad \com{continuous w.r.t. data}
\emp
\bmp{2in}
$\Delta u = 0 \mbox{ in } \Omega$ \quad \com{PDE ($u$ harmonic)}

\quad $u = f \mbox{ on } \Gamma$ \quad \com{Dirichlet BC}
\emp
\pig{.6in}{domain}

\pause
\vspace{-2ex}

Laplace fundamental soln: $\Phi(\xx,\yy) = \frac{1}{2\pi}\log \frac{1}{r}$ \;
\com{where $r := \|\xx-\yy\|$}
%\pig{.5in}{fundsol}
\hfill
\pig{.6in}{fs_lap}

\vspace{-3ex}

\quad obeys $-\Delta_\xx \Phi = -\Delta_\yy \Phi = \delta_\xx$
\quad \com{$\Phi$ harmonic except unit point-mass at $\mbf{0}$}

\quad\com{notation: $\nn$ points outwards, $\|\nn\|=1$, $u_n := \nn\cdot\nabla u$}

Green's 2nd identity: $\int_\Gamma vu_n - v_n u \, ds = \int_\Omega v\Delta u - (\Delta v)u \, d\yy$ \hfill \com{calculus}

\pause

\quad\com{warm-up: set $u = $ BVP soln, $v\equiv 1$, G2I becomes $\int_\Gamma u_n ds \, - 0 = 0 - 0$: so $u$ has zero {\em flux}}

\pause

\bmp{3.3in}
\quad\com{Now some fun: fix ``target'' $\xx\in\Omega$, let $v=\Phi(\xx,\cdot)$, G2I gives:}

Green's representation formula:
%\com{``GRF'', interior version, for any $\xx\in\Omega$:}

\hg

$ \int_\Gamma \Phi(\xx,\yy) u_n(\yy) - \frac{\partial \Phi(\xx,\yy)}{\partial \nn_\yy} u(\yy) \, ds_\yy =
u(\xx)$ \com{ \; for $\xx\in\Omega$}

\quad \com{recovers soln from ``Cauchy data'' $(u,u_n)|_\Gamma$}
  %  Works for all PDEs this week}

\hng

\quad\com{also versions for Helmholtz, Stokes, Maxwell,\dots}
% our first layer pots

% HW: show GRF for Helmholtz, assuming Phi is FS  (k^2 terms cancel)

\emp
\pig{.6in}{domainGRF}
\hfill
\pig{.6in}{dipole_lap}

%\com{reconstruct harmonic func from bdry data %$u_\Gamma$, $u_n|_\Gamma$ alone}

\end{noframe}


% LLLLLLLLLLLLLLLLLLLLLLLLLLLLLLLLLLLLLLLLLLLLLLLLLLLLLLLLLLLLLLLLLLLLLLLLLLLLLLLLLLLL
\begin{noframe}\ft{Layer potentials and their jump relations}

%\vspace{-1ex}

  \bmp{4in}
  Representations of harmonic functions off a curve $\Gamma$:
\;\com{``density'' $\sigma$}
  
  \sg

  Single-layer potential
  $(\Srep\sigma)(\xx) := \int_\Gamma \Phi(\xx,\yy) \sigma(\yy) ds_\yy$
  \;
    \com{charge sheet}
  \emp
\hfill \pig{.6in}{slp_lap}

  \bmp{4in}
  Double-layer potential
$(\Drep\sigma)(\xx) := \int_\Gamma \frac{\partial \Phi(\xx,\yy)}{\partial \nn_\yy}
  \sigma(\yy) ds_\yy$
  \com{dipole sheet}

  \vspace{4ex}

  \mbox{}
  \emp
\hfill \pig{.6in}{dlp_lap}

\vspace{-4ex}
\pause

\bmp{1.9in}
\com{interior (-) / exterior (+) limits:}
\emp
\bmp{2.5in}
\com{
  $u^{\pm}(\xx) := \lim_{h\to 0^+} u(\xx \pm h\nn_\xx)$    \quad 
}

\com{
  $u_n^{\pm}(\xx) := \lim_{h\to 0^+} \nn_\xx \cdot \nabla u(\xx \pm h\nn_\xx)$
 }
\emp

{\large Jump relations:}

\vspace{-5ex}

\bea
(\Srep\sigma)^\pm =&  S\sigma
 & \mbox{\com{$S$ (Roman font) means {\em restriction} of $\Srep$ to $\Gamma$: a bdry int.\ op.}}
\nonumber\\
(\Drep\sigma)^\pm =&  (D\pm I/2) \sigma
& \mbox{\com{jump in potential equal to $\sigma$; \; $D$ restriction to $\Gamma$ in P.V. sense}}
\nonumber \\
(\Srep\sigma)_n^\pm =&  (D^T\mp I/2) \sigma
& \mbox{\com{jump in normal derivative}} % equal to the charge {\em density} $\sigma$}}
\nonumber \\
(\Drep\sigma)_n^\pm =&  T \sigma
& \mbox{\com{$T$ hypersingular, kernel $\partial^2 \Phi(\xx,\yy) / \partial \nn_\xx \partial \nn_\yy \sim 1/r^2$}}
\nonumber
\eea

\vspace{-1ex}

\rb $D$ smooth kernel on smooth $\Gamma$, while $S$ always log (weakly) singular

% cptness?

%\rb Middle two JRs have $I/2$ identity terms which ...

\sg
\pause

Recap GRF in LP notation:
\hfill $u$ harmonic in $\Omega$
$\;\;\Rightarrow\;\;$
$\Srep u_n^- - \Drep u^- = u$ in $\Omega$

%happens to be zero outside (extinction)?



\end{noframe}




\begin{noframe}\ft{Converting BVP to BIE and solving}

\vspace{-1ex}
  
  \bmp{2in}
  Say wish to solve interior

  Dirichlet Laplace BVP:
  \emp
  \bmp{1.5in}
$\Delta u = 0 \mbox{ in } \Omega$ \quad \com{PDE}

\,$u^- = f \mbox{ on } \Gamma$ \quad \com{BC}
\emp
\only<1-5>{\pig{.6in}{domainplain}}%
\only<6->{\pig{.6in}{domainyt}}
% note the % making on same line needed to prevent mvmt btw two figures

\pause

Pick {\bf representation}: $u=\Drep \sigma$,\;
%for some unknown $\sigma$
look up its {\bf JR} for BC: $u^- = (D-I/2)\sigma$

\sg
\pause

\quad Insert the BC to get BIE: \qquad $(I - 2D)\sigma = -2f$  \hfill \com{scaled to 2nd kind form}

\quad \com{This shows: let $\sigma$ solve BIE, then $u=\Drep\sigma$ solves BVP (i.e., no spurious solns)}

\com{\dr{But how know {\em all} solns $u$ of this form?}}
%(GRF needed {\em two} LPs \frownie{}.)}
  \pause
  \com{\;Fred.\ Alt.: BIE {\em has} soln $\forall f$! BVP \& BIE \dr{equivalent} \smiley{}}

%\quad\com{\dots but what if there were a $u$ {\em not} in the form $\Drep\sigma$ for any $\sigma?$ Worry that GRF needed 

\quad \com{(had we picked $u=\Srep \sigma$, would get 1st kind, poorly conditioned but can have its uses)}

\pause

% *** cut this line (say it) to open space?
\com{Above BIE expressed on $\Gamma$ using arc-length measure $ds_\yy$. Usually not how $\Gamma$ described\dots}

\pause
{\bf Parameterize} the bdry $\yy(t)$ \hfill \com{$\yy:\R\to\R^2$, $2\pi$-periodic, $\Gamma = \{\yy(t): t\in[0,2\pi) \}$}

\quad\com{change variable $ds_\yy = \|\yy'(t)\|dt$ \quad abuse notation $\sigma(t) = \sigma(\yy(t))$}

\sg
\pause

Get 1D IE: \hfill  $\sigma(t) - 2 \int_0^{2\pi} \frac{\partial \Phi(\yy(t),\yy(s))}{\partial \nn_{\yy(s)}} \sigma(s) \|\yy'(s)\| ds = -2f(t),  \;\; t\in[0,2\pi)$
  
\quad \com{familiar form $(I+K)\sigma = -2f$, \; with kernel
  $k(s,t) = \frac{-2}{2\pi} \frac{\nn_{\yy(s)}\cdot(\yy(t)-\yy(s))}{\|\yy(t)-\yy(s)\|^2}\|\yy'(s)\|$}

\hfill \com{formula on diagonal: $k(t,t) = \lim_{s\to t} k(t,s) = \kappa(t)/2\pi$, \; $\kappa$ curvature of $\Gamma$ \; (check!)}

% HW: show that k(t,s) = const, for Gamma the unit circle.
% HW: get the analytic solution for the 1D integral equation for Gamma the unit circle for
%    i) Dirichlet BC f=1, [Hint: you should get sigma=-1],   then ii) arbitrary f.
% HW check the diagonal limit above [Hint: l'Hopital's rule]

\pause
\sg

Now Nystr\"om discretize with PTR, solve lin.\ sys.\ for $\bm\sigma:=\{\sigma_j\}_{j=1}^N$

Finally evaluate soln: $u(\xx) = (\Drep\sigma)(\xx)
\,\stackrel{\tbox{PTR}}{\approx}\,
\sum_{j=1}^N
\frac{\nn_{\yy(t_j)}\cdot(\xx-\yy(t_j))}{2\pi\|\xx-\yy(t_j)\|^2}\|\yy'(t_j)\|w_j \sigma_j$

%    see ``speed weights''    
    
\end{noframe}


\begin{frame}[fragile] \ft{Interior Laplace Dirichlet BVP solve demo}

\vspace{-5ex}
  
\hfill\com{{\tt demo\_lapintdir.m}}

\begin{matc}
a=0.7; b=1.0;                                                   % shape params (note a=1,b=0 unit circle)
Y = @(t) [a*cos(t)+b*cos(2*t); sin(t)];                         % kite parameterization y(t)
Yp = @(t) [-a*sin(t)-2*b*sin(2*t); cos(t)];                     % y', analytic
Ypp = @(t) [-a*cos(t)-4*b*cos(2*t); -sin(t)];                   % y'', analytic
N = 100;
t = 2*pi/N*(1:N); w = 2*pi/N*ones(1,N);                         % PTR nodes & weights
y = Y(t);                                                       % bdry nodes, 2-by-N
n = [0 1;-1 0]*Yp(t); speed = sqrt(sum(n.^2,1)); n = n./speed;  % bdry normals
kappa = -sum(Ypp(t) .* n,1)./speed.^2;                          % bdry curvatures
r1 = y(1,:)'-y(1,:); r2 = y(2,:)'-y(2,:);                       % matrix of r=x-y (two vec cmpnts)
A = (-1/pi)*(n(1,:).*r1 + n(2,:).*r2) ./ (r1.^2+r2.^2);         % off-diag (-1/pi) n.r/r^2
A(diagind(A)) = kappa/(2*pi);                                   % overwrite diag elements
A = eye(N) + A*diag(speed.*w);                                  % note Id gets no "speed weights"
uex = @(x) ([1 0]*x) .* ([0 1]*x-0.3);                          % test u(x) = x_1(x_2-0.3), not symmetric!
f = @(t) uex(Y(t));                                             % read off its Dirichlet data
rhs = -2*f(t)';
sigma = A\rhs;                                                  % solve. Leave u = D.sigma eval to reader
\end{matc}

\pause
\pig{1.4in}{lapintdir}
\pig{0.9in}{lapintdir_conv}
\hfill
\pause
\pig{1.3in}{lapintdir_err}
\bmp{0.9in}
\com{error: ``5h'' rule.}         % *** to clarify: 5h wide strip add label to fig?


\com{Note: special}

\hng
  
\com{quadratures}

\hng

\com{can fix near $\Gamma$}

\hng

\who{Helsing, QBX...}
\emp
\vspace{-0.5ex}
\pause

Debug: $\sigma\equiv -1 \; \Rightarrow \; u\equiv 1$, then test data from (generic!) soln $u$, and\dots
\ben
\item check/plot $\nn$, $\kappa$. First test unit circle!
 \item check Nystr\"om matrix smooth at diag \com{({\em before} add $I$)}
\een
% *** I'm not sure this list idea is working - save it for final tweaks

\end{frame}
  



% *** SAVE FOR END?
\begin{frame}\ft{Indirect vs direct formulations}

\vspace{-3ex}
  
  \hfill\com{using Laplace interior Dirichlet BVP}

\sg
  
So far ``indirect'' BIE: pick representation (eg $u=\Drep\sigma$), get BIE from JRs

%\com{prove BVP $\leftrightarrow$ BIE equivalence}

\pause
\sg

Alternative is ``direct'': take limit of GRF on $\Gamma$, rearrange to get BIE:

\sg

\quad\com{GRF $u = \Srep u^- - \Drep u_n^-$ \; $\stackrel{\text{JRs}}{\longrightarrow}$ \;
  $u_n^- = (D^T+I/2)u_n^- - Tu^-$ \;
$\stackrel{\text{BC}}{\longrightarrow}$ \;   $(D^T-I/2)u_n^- = Tf$}

\quad\com{Needs hypersingular apply \frownie{}. Then solve BIE for $u_n^-$, eval $u$ via GRF (needs two LP evals)}

\vg
\pause

Notice 
BIO $(D^T-I/2)$
{\bf adjoint}
of that for indirect $(D-I/2)$

\quad\com{generally true. So, spectra the same, thus iterative convergence rates too}

\vg
\pause

%Compare pros and cons:

%\sg

\begin{tabular}{l|l}   %  Table: direct vs indirect pros/cons
  Indirect BIE & Direct BIE \\
  \hline
  unknown density (unphysical) & unknown is physical\\
  RHS is plain data & RHS needs BIO apply to data\\
  eval the representation (may be simpler) & eval the GRF% (SLP \& DLP)
\end{tabular}

%Various factors; We tend to prefer indirect

\vg

\rb indirect: more flexibility, but need math to prove equivalence to BVP

\rb accuracy differences for domains with corners \who{Hoskins--Rachh\dots}

\end{frame}


%%%%%%%%%%%%%%%%%%%%%%%%%%%%%%%%%%%%%%%%%%%%%%%%%%%%%%%%%%%%%%%%%%%%%%%%%%%%%%%%%%%%%%%%%%
\begin{noframe}\ft{Indirect 2nd-kind BIE for Neumann, exterior}
  %    \vspace{-5pt}

  % *** PS the right tool for this would be 4 separate \bmp{}...\emp blocks:
        \begin{tabularx}{\textwidth}{ X | X }
            \vspace{-5pt} \centering recap: {\color{red} Laplace int.\ Dir.} \vspace{5pt} & \vspace{-5pt} \centering\arraybackslash {\color{red} Laplace int.\ Neu.} \\
        \bmp{1.5in}
              $\Delta u = 0 \mbox{ in } \Omega$ 

              $u^- = f \mbox{ on } \Gamma$
        \emp & \bmp{1.5in}
              $\Delta u = 0 \mbox{ in } \Omega$ 

              $u_n^- = g \mbox{ on } \Gamma$
        \emp \\
        uniqueness, existence $\forall f$ & require $\int_{\Gamma} g ds  = 0$\\
        & unique only up to a const. \\
        \rb $u = \mathcal{D}\sigma$ \qquad  \com{rep.} &
        \mbox{\rb $u = \mathcal{S}\sigma$ \hspace{.3in} \com{$\mbox{}_\swarrow$kernel${\equiv}1$, kills nullspace}}
        \\
        $(D - I/2)\sigma = f$ \quad \com{BIE: well-cond.} & $(D^T + I/2 + 11^T)\sigma = g$ \;\com{well-cond.}\\
        \hline
        \end{tabularx}
        
        \pause       % apparently can't be done inside tabularx env :(
        \vspace{-1pt}
        
        \begin{tabularx}{\textwidth}{ X | X }
        \vspace{-5pt} \centering {\color{red} Laplace ext.\ Dir.} \vspace{5pt} & \vspace{-5pt} \centering\arraybackslash {\color{red} Laplace ext.\ Neu.} \\
        \bmp{2.2in}
              $\Delta u = 0 \mbox{ in } \mathbb{R}^2 \backslash \overline{\Omega}$ 


              $u^+ = f \mbox{ on } \Gamma$   %\qquad \com{$\mbox{}_{\swarrow}$net charge zero} % too much info
        \emp
        &
        \bmp{2.2in}
              $\Delta u = 0 \mbox{ in } \mathbb{R}^2 \backslash \overline{\Omega}$ 


              $u_n^+ = g \mbox{ on } \Gamma$
        \emp
        \\
        $u_{\infty} := \lim_{\|\xx\|\to \infty} u(\xx)$ exists
        &
        require $\int_{\Gamma} g ds  = 0$ and $u_{\infty} = 0$
        \\
        uniqueness, existence $\forall f$
        &
        unique\\
        \rb $u = \mathcal{D}\sigma + \int_\Gamma \sigma ds$ \quad \com{modified rep.}
        &
        \rb $u = \mathcal{S}\sigma$ \\
        $(D+ I/2 +11^T)\sigma = f$ \quad \com{well-cond.}
        &
        $(D^T - I/2)\sigma = g$ \quad \com{well-cond.}
    \end{tabularx}

        % HW:
        % int Neu:
        % a) prove the Lemma: Let L have kernel L(t,s)\equiv 1, and let $\int_\Gamma g = 0$.
        % Then any soln to (D^T + I/2 + L)\sigma = g also solves (D^T + I/2)\sigma = g.
        % [hint: (D+1/2)1 = 0 by GRF, really zero-flux]
        % b) prove the lemma: D^T + I/2 + L is injective, so that modified BIE well-cond.
        
        % HW:   ext Dir: prove D+I/2+L is injective.

\pause
\vspace{-2.3in}
\hspace{1.535in}\begin{tikzpicture}
  \node[double arrow, fill=red!50, anchor=base, rotate = 45, align=center,text width=2in] {complementary BVPs};
\end{tikzpicture}

\pause
\vspace{-1.68in}
\hspace{1.535in}\begin{tikzpicture}
  \node[double arrow, fill=red!50, anchor=base, rotate = -45, align=center,text width=2in] {complementary BVPs};
\end{tikzpicture}

\pause
\vspace{0.7in}

  \ben
\setcounter{enumi}{2}   % number you want -1
\item Exterior: don't test with $u=\log r$! \; \com{why not? BVPs enforce zero net charge}
%\item Neumann: check contours $\perp \Gamma$  % killed since unhelpful unless g=0 or u^tot, too hard to explain here.
\een



\end{noframe}




\begin{frame}\ft{Helmholtz --- introduction and connection to Maxwell}

  $(\Delta + \omega^2)u = 0$ \hfill\com{time-harmonic scalar waves}

  \sg
    comes from scalar wave equation $\Delta u - u_{tt} = 0$ when $u(\xx,t) = u(\xx)e^{-i\omega t}$
    
    $\omega$ is the wavenumber \com{spatial frequency, related to wavelength via $\lambda = 2\pi/\omega$}

 \sg 
    Also used for Maxwell's equations in cylindrical symm (z-invariance):
    \sg

    \bmp{0.61\textwidth}

        \quad\com{1. Assume $\mathbf{E}, \mathbf{H}(x, y, z) = \mathbf{E}, \mathbf{H}(x, y)$}
        
        \quad\com{2. Write Maxwell's eqs: $\nabla \times \mathbf{E} = i\omega \mu \mathbf{H}$, $\nabla \times \mathbf{H} = -i\omega \varepsilon \mathbf{E}$,}
       
       \quad\com{3. Notice only $E_z$, $H_z$ are indep $\to$ 2 polarizations, TE or TM: $E_z = 0$, $H_z = 0$ resp.} 

       \quad\com{4. Choose TE and let $u := H_z$, then: $\mathbf{H} = (0, 0, u)$, $\mathbf{E} = \tfrac{1}{i\omega\varepsilon}(\partial_x u, -\partial_y u, 0)$, and $(\Delta + n^2\omega^2)u = 0$ with $n^2 = \varepsilon \mu $}
    \emp
    \hfill
    \pig{1.6in}{TE-TM-formalism}
\sg

    Dirichlet BC in TE formalism = PEC \quad \com{perfect electric conductor; $\mathbf{E}$ $\perp$ to surface} 

\end{frame}

\begin{frame}\ft{Helmholtz --- scattering formalism}
    \bmp{0.8\textwidth}
    Split total potential into incident \com{(known)} and scattered \\ \com{(unknown)} parts,
  $u^\tbox{tot} = u^\tbox{inc} + u$
    \emp
    \hfill
    \pig{0.8in}{scattering-formalism}


    BVP for $u$: 

    \quad $ (\Delta + \omega^2)u = 0 \quad \text{in } \mathbb{R}^d \backslash \overline{\Omega} \quad \text{\com{PDE}}$

    \quad $ u = -u_i \quad \text{on } \Gamma \quad \text{\com{Dirichlet BC, $u_n = -(u_i)_n$ for Neumann}}$

    \quad $ \lim_{r \to \infty} \left( \frac{\partial u}{\partial r} - i\omega u \right) = 0 \;\text{\com{$r := |\xx - \yy|$, Sommerfeld radiation condition for uniqueness}}$

    \sg
    \bmp{0.7\textwidth}
    Fundamental solution $ \Phi(\xx, \yy) = \frac{i}{4}H_0^{(1)}(\omega|\xx - \yy|)$

    \quad \com{Asymptotics: $\lim_{r \to 0} \Phi(r) = \frac{1}{2\pi} \log \frac{1}{r} + \bigO(1)$ }

    \quad \com{
        \hspace{46pt} $ \lim_{r \to \infty}\Phi(r) = \sqrt{\frac{2}{\pi r}} e^{i(r - \nu\pi/2 - \pi/4)} + \bigO(r^{-1}) $ }

    \quad \com{Same singularity as Laplace $\to$ same JRs!}
    \emp\hfill
    \pig{1.2in}{helmholtz-fundsol}

    \bmp{1.2in}
    Layer potentials
    \emp
    \hspace{0.3in}
    \pig{0.6in}{slp_helm.pdf}\quad\com{SLP}
    \hspace{0.1in}
    \pig{0.6in}{dlp_helm.pdf}\quad\com{DLP}

\end{frame}

\begin{frame}\ft{Helmholtz --- interior resonances and how to avoid them}

    Try the ext Dir BVP with $u = \Drep \sigma $ 
    \com{$(\Delta + \omega)^2 u = 0 $ in $\mathbb{R}^2\backslash\overline{\Omega}$, $u = -u_i$ on $\Gamma$, SRC for $u$ }
\sg

    Observe that for some $\omega$, condition $\#$ of BIE blows up, not always solvable
    
    \quad\com{Why? Suppose $\phi \not \equiv 0$ s.t.  \bmp{1.2in} $\begin{cases} (\Delta + \omega^2)\phi = 0 \quad \text{in } \Omega  \\ \phi_n = 0 \quad \text{on } \Gamma \end{cases}$ \emp \hfill \bmp{1.8in} $\phi$ is interior Neumann eigenfunction with eigenvalue $\omega^2$\emp }

    \quad \com{Then by (interior) GRF (same as for Laplace), $\Srep \cancel{\phi_n|_{\Gamma}} - \Drep \phi|_{\Gamma} = u \quad \text{in } \Omega$.}

    \quad   \com{Take $\mathbf{x} \to \Gamma^{-}$ and use JR: $(-D - I/2)\phi|_{\Gamma} = \phi_{\Gamma}$, i.e. $(I + 2D)\phi|_{\Gamma} = 0.$}

    \quad   \com{Since $\phi|_{\Gamma}$ was nontrivial (otherwise $\phi = 0$ by GRF), nullity of $I + 2D > 0$, i.e.\ singular, by FA not solvable $\forall f$ ($u_i$).}

    \quad   \com{We made use of the \textbf{complementary BVP} (int Neu), this is an ``internal resonance".  }
\sg

    Fix: $u = (\Drep - i\eta\Srep)\sigma$  \bmp{3.2in} \com{combined field integral eq (CFIE), same $\#$ unknowns, new kernel}

    \com{ext Dir BIE becomes $(I + 2D - 2i\eta S)\sigma = -2u_i \quad \text{on }\Gamma$}   \emp
\sg

    \quad \com{Proof: Let $\tau$ solve $(I/2 + D - i\eta S)\tau = 0$, wish to show $\tau = 0$.}

    \quad \com{From $\tau$ construct potential $v:= (\Drep - i\eta \Srep)\tau,$ then $v^{+} = 0$ by construction.}

    \quad \com{Then $v = 0$ in $\mathbb{R}^2 \backslash \overline{\Omega}$ by uniqueness of the complementary BVP (ext Dir)  }

    \quad \com{Then $v_n^{+}$ on $\Gamma$, and by JRs and Green's 1st thm (exercise for the reader \smiley), $\tau = 0$.}

\end{frame}

\begin{frame}\ft{Helmholtz --- Dirichlet demo}
\vspace{-5ex}
\hfill\com{{\tt demo\_helmextdir.m}}

    \vspace{5ex}
    Solve the Helmholtz ext Dir BVP with the $u = \Drep \sigma$ repr, $u_i$ plane wave
    \sg 

    Diagonal limit for Nystr\"{o}m matrix $k(t, t)$ same as Laplace
    \sg 

    PTR with $N$ nodes, test via self-convergence \com{What's the conv.\ rate?} \com{ Why $N^{-3}$?}

%  Solve BVP for $u$ via PTR + Nystr\"om, with new diag limit for $k(t,t)$,
%  show $1/N^3$ convergence if use naive PTR with correct diag limit (see M126 HW?)
\sg

\pause
\vspace{5pt}
\pig{1.4in}{helmextdir}
\hfill
\pig{1.3in}{helmextdir_conv}
\hfill
\pause
\pig{1.3in}{helmextdir_err}

  
  \ben
\setcounter{enumi}{3}
  \item Debug BVP with known data from a radiative soln \com{sources inside $\Omega$}
  \item Without analytic soln, test either via self-convergence or conserved physical qty \com{e.g.\ optical theorem, or no net QM flux over closed curve $C$ containing no sources or sinks, 0 = $\Im\left( \int_{C} \bar{u}u_n ds \right)$ \who{Agocs, Barnett '23}}
\een

%(don't demo CFIE since requires S w/ log-singularity).

\end{frame}

\begin{frame}\ft{Helmholtz -- transmission BVP}

  
%  Matching? (more effort, needs $\Srep \sigma + \Drep \tau$ \dots)
  
%  difference kernels at most log-singular.

\bmp{0.7\textwidth}
    If different refractive index $n$ in $\Omega$ than outside, use usual splitting $u^\tbox{tot} = u^\tbox{inc} + u$
 
    \com{can always scale such that one is $n = 1$}
    
    \com{inc wave only on outside, e.g.\ $u_i = \begin{cases} 0 & \text{in }\Omega \\ e^{i \mathbf{k}\cdot \xx} & \text{in }\mathbb{R}^2\backslash \overline{\Omega} \end{cases}$, $\mathbf{k} = \begin{bmatrix} \omega\cos\theta \\ \omega\sin\theta \end{bmatrix}$  }


\emp
\hfill
\pig{1.0in}{helm-transmission}

BVP for $u$: 

\quad $ (\Delta + \omega^2)u = 0 \quad \text{in } \mathbb{R}^d \backslash \overline{\Omega} \quad \text{\com{PDE outside}}$

\quad $ (\Delta + n^2\omega^2)u = 0 \quad \text{in } \overline{\Omega} \quad \text{\com{PDE inside}}$

    \quad $ [u] = -u_i \quad \text{on } \Gamma \quad \text{\com{$[u] := u^{+} - u^{-},$ continuity of $ u^\tbox{tot}$ }}$

\quad $ [u_n] = -(u_i)_n \quad \text{on } \Gamma \quad \text{\com{continuity of $ u^\tbox{tot}_n$ }}$

\quad $ \lim_{r \to \infty} \left( \frac{\partial u}{\partial r} - i\omega u \right) = 0 \quad\text{\com{SRC outside}}$

\sg
    Formulate as sys of integral eqs \quad \com{Rokhlin--M\"{uller} scheme,} \who{M\"{u}ller '69, Rokhlin '83}

  \com{$u = \begin{cases} \Srep^{(n\omega)}\sigma + \Drep^{(n\omega)}\tau & \text{in }\Omega \\ \Srep^{(\omega)}\sigma + \Drep^{(\omega)}\tau & \text{in }\mathbb{R}^2 \backslash \Omega\end{cases} $}

  \com{$\begin{bmatrix} [u] \\ [u_n] \end{bmatrix}  = \left( \begin{bmatrix} I & 0 \\ 0 & I \end{bmatrix} + \begin{bmatrix} D^{(\omega)} - D^{(n\omega)} & S^{(n\omega)} - S^{(\omega)}\\ T^{(\omega)} - T^{(n\omega)}& D^{(n\omega)\ast} - D^{(\omega)\ast}\end{bmatrix} \right) \begin{bmatrix} \tau \\ -\sigma \end{bmatrix}$ \quad $T$ is hypersingular operator }

  \com{ ...but $T^{(\omega)} - T^{(n\omega)}$ is at most log-singular! \smiley\; (Show via asymptotics of $H_n^{(1)}$)}

\end{frame}

\begin{frame}\ft{Helmholtz -- high-order accuracy}


  Spectral accuracy Nystr\"om for log-singular kernels: possible, but beyond today
    \sg

    \com{Divide bdry into panels instead of global set of nodes, adaptive panel sizes \& quadrature rules}

    \com{Kernel-split: decompose kernel $G(\xx, \yy) = \underbrace{G^S(\xx, \yy)}_{\text{smooth}} + \underbrace{G^L(\xx, \yy)\log|\yy - \xx|}_{\text{log singularity}} + \underbrace{G^C(\xx, \yy) \frac{(\yy - \xx) \cdot \mathbf{n}}{|\yy - \xx|^2}}_{\text{Cauchy singularity}} $}

    \com{Product integration: target-specific quadrature rules, e.g.\ $\int_{\Gamma}f(\xx, \yy)\log|\xx - \yy| ds_{\yy} \approx \sum_{j = 1}^N f(\xx, \yy_j) w_j^L(\xx)  $  \who{Helsing, Holst, '15}, \who{Kress}, \ldots} 

    \com{Generalized Gaussian quadrature \who{Bremer}}
    \sg

    Close evaluation: target close to bdry %kernel-split \who{Helsing, Ojala '08}, QBX, etc.\

    \com{Kernel-split approach}

    \com{QBX: quadrature by expansion \who{Kloeckner, Barnett, Greengard, O'Neil '13}, \who{Epstein, Greengard, Kloeckner '13}}

    \sg 

  See also libraries: chunkie, BIE2D, etc.\

  \end{frame}



%\begin{frame}\ft{More debug ideas}
%
%TO DISCARD
%  
%%    \newcounter{enumTemp}      % only do this once
%%    \setcounter{enumTemp}{\theenumi}
%  
%
%Other tests:
%
%\ben
%\setcounter{enumi}{4}
%\item Test SLP \& DLP evaluators via GRF for any harmonic $u$ in $\Omega$
%\een
%
%\end{frame}




\begin{frame}\ft{Summary}

% *** TO EDIT WITH FRUZSINA:

  Covered BIE basics for smooth curves with global quadrature:

\sg
  
  \rb Well-posed Laplace \& Helmholtz BVPs\hfill
 \com{exterior need condition as $\|\xx\|\to\infty$}
  
  \rb Choosing representation \hfill \com{to get 2nd kind BIE if poss., equivalent to BVP if poss.}

  \qquad \com{Can switch interior/exterior, Laplace/Helmholtz/etc, via simple code changes}
  
  \rb Nystr\"om discretization \hfill \com{high-order/spectral convergence, if poss.}

  \rb Build/debug codes via well-chosen sequence of test cases
  \hfill\com{also for libraries}

  \sg
  
  \dr{\small practise! write out theory yourself + try HW exer.\ in repo + run demos}

  
%\rb Route: representation $\stackrel{\text{JRs}}{\longrightarrow}$ BIE
%$\stackrel{\text{}}{\longrightarrow}$ lin sys

\pause \vg

Useful 2D tools we did not yet cover:
\hfill\com{in libraries, eg {\tt chunkie}, {\tt BIE2D}}

\sg

\rb panel (composite) quadratures \hfill\com{essential for adaptivity}

\rb high-order quadratures for log-singular kernel \hfill\com{SLP, Helmholtz, etc}

\rb special quadratures for evaluation close to the curve

 \qquad\com{some need interpolation of $\sigma(t)$ off the nodes $t_j$, some not}

 \rb corners, open arcs, slits, multi-material junctions

%\rb Nystr\"om is not only way to discretize: Galerkin, collocation
  
%\qquad \com{but: simplest and no less accurate than others}


  
%  \rb Nystr\"om discretization gets $\sigma(t_j)$ \com{interpolate from them to other $t$}
%  \rb Fancier quadratures needed for singular kernels and/or close eval


  
\end{frame}


\begin{frame}\ft{Resources}  % RRRRRRRRRRRRRRRRRRRRRRRRRRRRRRRRRRRRRRRRRRRRRRR

Many numerical analysis (mathematics heavy). Somewhat accessible:

\hg

\rb {\em Linear Integral Equations}, R. Kress, (1999, Springer). Ch. 6 \& 12.

\hg

\rb {\em The Numerical Solution of Integral Equations of the Second Kind},
K. E. Atkinson, (1997, CUP).

% *** add Sauter-Schwab

% *** Ladyzhenskaya, Hsiao--Wendland for Stokes?


\sg

Fewer on the practical/tutorial side, few with modern devels:

\hg

\rb ``High-order accurate methods for Nystr\"om discretization of integral equations on smooth curves in the plane'', S Hao, AH Barnett, PG Martinsson, P Young.
{\em Adv. Comput. Math.} {\bf 40}, 245--272 (2014).

\hfill \com{focuses on quadrature for logarithmic singularities,
  eg SLP, Helmholtz}

\hg

\rb \url{https://users.flatironinstitute.org/~ahb/BIE/}

\hg

\rb \url{https://github.com/ahbarnett/BIEbook} \hfill \com{in progress\dots}

\end{frame}


\end{document}
