\documentclass[t]{beamer}
\usepackage[utf8]{inputenc}
\usepackage[swedish,english]{babel}

% \usetheme{CCM}
\usepackage{../../templates/beamerthemeCCM}
\usepackage{epstopdf}  % handle EPS
\usepackage{bm}
\usepackage{mathtools}
\mathtoolsset{showonlyrefs}

% alex macros
\input{../../templates/rgb}
\newcommand{\ft}[1]{\frametitle{#1}}
\newcommand{\bi}{\begin{itemize}}
\newcommand{\ei}{\end{itemize}}
\newcommand{\ben}{\begin{enumerate}}
\newcommand{\een}{\end{enumerate}}
\newcommand{\be}{\begin{equation}}
\newcommand{\ee}{\end{equation}}
\newcommand{\bea}{\begin{eqnarray}}
\newcommand{\eea}{\end{eqnarray}}
\newcommand{\bc}{\begin{center}}
\newcommand{\ec}{\end{center}}
\newcommand{\mbf}[1]{{\bm #1}}           % requires bm package
\newtheorem{thm}{Theorem}
\newcommand{\ig}[2]{\includegraphics[#1]{#2}}
\newcommand{\tbox}[1]{{\mbox{\tiny #1}}}
\newcommand{\who}[1]{{\scriptsize \textcolor{darkgreen}{(#1)}}}  % cite
\newcommand{\com}[1]{{\scriptsize \textcolor{purple}{#1}}}      % comment
\newcommand{\co}[1]{{\scriptsize \tt #1}}          % code
\newcommand{\vg}{\vspace{2ex}}
\newcommand{\sg}{\vspace{1ex}}
\newcommand{\hg}{\vspace{0.5ex}}
\newcommand{\dr}[1]{{\color{darkred}#1}}    % for bullets
\newcommand{\rb}{\ensuremath{\textcolor{darkred}{\bullet\;}}\ }
\newcommand{\bmp}[1]{\begin{minipage}{#1}}
\newcommand{\bmpt}[1]{\begin{minipage}[t]{#1}}
\newcommand{\emp}{\end{minipage}}
\newcommand{\pig}[2]{\bmp{#1}\includegraphics[width=#1]{#2}\emp} % mp-fig, nogap
\newcommand{\pigm}[3]{\bmp{#1}\href{#3}{\includegraphics[width=#1]{#2}}\emp} % w/ movie
\newcommand{\ora}[1]{{\color{orange} #1}}
\newcommand{\gre}[1]{{\color{green} #1}}
\newcommand{\yel}[1]{{\color{yellow} #1}}
\newcommand{\sr}[1]{{\scriptsize #1}}
\newcommand{\eps}{\epsilon}
\newcommand{\qqquad}{\qquad\qquad}
\newcommand{\qqqquad}{\qqquad\qqquad}
\newcommand{\R}{\mathbb{R}}
\newcommand{\Z}{\mathbb{Z}}
\newcommand{\N}{\mathbb{N}}
\newcommand{\emach}{\epsilon_\tbox{mach}}
\DeclareMathOperator{\im}{Im}
\DeclareMathOperator{\re}{Re}
\newcommand{\bigO}{{\cal O}}
\newcommand{\pO}{{\partial\Omega}}
\newcommand{\xx}{\mbf{x}}
\newcommand{\yy}{\mbf{y}}
\newcommand{\uu}{\mbf{u}}
\newcommand{\nn}{\mbf{n}}
\newcommand{\Srep}{{\cal S}}
\newcommand{\Drep}{{\cal D}}


\title{Singular quadrature for layer and volume potentials}

\date{Computational Tools for PDEs with Complicated Geometries and Interfaces\\
  Day 3, 12 June 2024}
\author{\textbf{Ludvig af Klinteberg (Chair)}\inst{1}}
\institute{\inst{1} Mälardalen University (MDU), Västerås, Sweden}

\begin{document}

\begin{frame}
	\titlepage
\end{frame}

\begin{frame}[c]{Session plan}
  Start: 10.00
  \bigskip
  \begin{itemize}
  \item Introduction talk (30--40 min), Ludvig af Klinteberg
  \item Quick break
  \item Short talks (5--10 min each)
    \begin{itemize}
    \item Thomas G. Anderson, Rice
    \item David Krantz, KTH
    \item Zydrunas Gimbutas, NIST
    \item Bowei Wu, UMass Lowell
    \end{itemize}
  \item Group discussion
  \end{itemize}
  \bigskip  
  End: 12.30
\end{frame}

\begin{frame}{Layer Potentials}
  We represent solutions to BVPs as layer potentials
  \begin{align}
    u(\xx) = \int_\Gamma k(\xx, \yy) \sigma(\yy) ds_\yy
  \end{align}
  with $k$ some weakly singular kernel \com{e.g. Laplace 2D SLP $k(\xx,
    \yy) = \frac{1}{2\pi} \log\|\xx-\yy\|$}

  \sg
  Density $\sigma$ is solution to integral equation such as $(I+K)\sigma=f$ on $\Gamma$

  \pause
  \vg
  We will need to evaluate $u(\xx)$ in three different regimes:
  \begin{itemize}
  \item \textbf{Smooth:} $\xx$ far away from $\Gamma$
    \hfill \com{everything is easy here}
  \item \textbf{Singular:} $\xx$ on $\Gamma$
    \hfill \com{typically occurs when solving the integral equation}
  \item \textbf{Nearly singular:} $\xx$ arbitrarily close to $\Gamma$
    \hfill \com{surprisingly difficult}
  \end{itemize}
\end{frame}

\begin{frame}{Input geometry}
  Our input is a geometry parametrization
  $\Gamma = \{\yy(t) \::\: t \in D\}$ \com{line or surface}
  \begin{align}
    u(\xx) = 
    \int_\Gamma k(\xx, \yy) \sigma(\yy) ds_\yy
    =
    \int_D k(\xx, \yy(t)) \sigma(t) |J_\yy(t)|dt
  \end{align}
  \hfill\com{note $\sigma(t) = \sigma(\yy(t))$}

  \pause
  \vg
  Parametrization can be:
  \begin{itemize}
  \item Analytical
  \item In terms of basis functions (polynomials, spherical harmonics, ...)
  \item (Just points will do, because we can interpolate to basis functions)
  \end{itemize}
\end{frame}

\begin{noframe}{Layer Potential Discretization}
  
  We discretize with a quadrature rule $\{t_j, w_j\}$ for the \emph{smooth} regime,
  \begin{align}
    u(\xx) = 
    \int_\Gamma k(\xx, \yy) \sigma(\yy) ds_\yy
    \approx
    %\sum_{j=1}^N k(\xx, \yy_j) \sigma_j s_j w_j
    %=
    \sum_{j=1}^N k\left(\xx, \yy(t_j)\right) \sigma(t_j) |J_\yy(t_j)| w_j, 
  \end{align}

  \pause
  \begin{itemize}
  \item Underlying/smooth quadrature is our base discretization
  \item Can \emph{interpolate} surface quant. ($\sigma$, $\yy$, $\partial_t\yy$) from node data
    \pause
  \item Quadrature either \emph{global} or \emph{local} (composite / panel-based)
  \end{itemize}
  \begin{center}
    \begin{tabular}{ccc}
      \includegraphics[width=0.2\textwidth]{fig/starfish_trapz} &
                                                                  \hspace{3em}
      &
        \includegraphics[width=0.2\textwidth]{fig/starfish_panels}\\
      Periodic trapezoidal rule (PTR) & & Gauss-Legendre (G-L) panels \\
      \com{trigonometric interpolation} & & \com{polynomial interpolation} \\
      \com{error $\mathcal O(e^{-cp}), \quad N\sim p^{d-1}$} &
      &
        \com{error $\mathcal O(h^p), \quad N \sim \left(\frac{p}{h}\right)^{d-1}$} \\                                            
    \end{tabular}
  \end{center}
\end{noframe}
  
\begin{noframe}[c]{Global and local quadratures}
  \begin{center}
    \begin{tabular}{ccc}
      \includegraphics[height=0.2\textwidth]{fig/starfish_trapz}
      &
        \includegraphics[height=0.18\textwidth]{fig/stellarator_trapz}
      &
        %\includegraphics[height=0.18\textwidth]{fig/two_close_spheroids}
        %\includegraphics[height=0.2\textwidth]{fig/vesicles}
        \includegraphics[height=0.18\textwidth]{fig/spharm_geo}
      \\
      \com{PTR} & \com{Double PTR} & \com{PTR + G-L} \\
      \hline
      \vspace{-1em}
      \\
      \includegraphics[height=0.2\textwidth]{fig/starfish_panels}
      &
        \includegraphics[height=0.18\textwidth]{fig/stellarator_gl}
      &
        \includegraphics[height=0.18\textwidth]{fig/torus_triangulation}
      \\
      \com{G-L panels} & \com{G-L (or Cheb.) quads} & \com{Vioreanu-Rokhlin triangles}
      \\
      \includegraphics[height=0.15\textwidth]{fig/legpts}
      &
        \includegraphics[height=0.13\textwidth]{fig/tensorlegpts}
      &
        \includegraphics[height=0.15\textwidth]{fig/vr-triangle}
    \end{tabular}
  \end{center}
\end{noframe}

\begin{noframe}{What we need}
  The linear map that evaluates $u$ from $\sigma$ at a set of target points $\{\xx_i\}$,
   $\uu=A\bm{\sigma}$:
    \begin{align}
      u(\xx_i) = \sum_j a_{ij} \sigma_j,
    \end{align}
    \begin{itemize}
      \uncover<2->{
    \item For targets where smooth quadrature applies,
    \begin{align}
      a_{ij} = k\left(\xx_i, \yy(t_j)\right) |J_\yy(t_j)| w_j
    \end{align}
    \com{This we can compute $\forall \xx_i$ using fast methods (FMM)}
  }
  \uncover<3->{
  \item For other targets, we need something clever

    \com{Usually target-dependent: not FMM'able.}
    }
  \end{itemize}
  \includegraphics[height=0.20\textwidth]{fig/curve_nearfar}
  \includegraphics[height=0.25\textwidth]{fig/matrix_nearfar}
\end{noframe}

\begin{frame}{The problem}
  The kernel is composed of singularities and smooth parts, e.g.
  \begin{align}
    \text{2D: }
    k(\xx, \yy) &= k_0(\xx, \yy) +
                  \log\|\xx-\yy\| k_L(\xx, \yy) +
                  \frac{(\xx-\yy)\cdot\nn}{\|\xx-\yy\|^2} k_2(\xx, \yy) + \dots
    \\
    \text{3D: }
    k(\xx, \yy) &= k_0(\xx, \yy) +
                  \frac{k_1(\xx, \yy)}{\|\xx-\yy\|}  + 
                  \frac{(\xx-\yy)\cdot\nn}{\|\xx-\yy\|^3} k_3(\xx, \yy) + \dots
  \end{align}
  with $k_{(\cdot)}(\xx,\yy)$ smooth functions in $\yy$. \com{We can usually write down this split.}

  \pause
  \sg
  \begin{itemize}
  \item \textbf{Smooth regime:} $\|\xx-\yy\|$ varies slowly when $\xx$ far from $\Gamma$
  \item \textbf{Singular regime:} Weak (integrable) singularities as $\yy \to \xx$
  \item \textbf{Nearly singular regime:} $k(\xx,\yy)$ ''sharply
    peaked'' when $\xx$ close to $\Gamma$
  \end{itemize}
  \begin{center}
    \includegraphics[width=0.6\textwidth]{fig/kernel_curve}
  \end{center}
  \com{TODO: Fix better plot}
\end{frame}

\begin{frame}{Kernels on Complex Form}
  Things get simpler in 2D if we use complex variables.

  Let $(\zeta, \tau)$ correspond to $(\xx, \yy)$ in $\mathbb C$, then the Laplace DLP is
  \begin{align}
    \int_\Gamma \frac{(\xx-\yy)\cdot\nn}{\|\xx-\yy\|^2} \sigma(\yy) ds_\yy
    =
    \im \int_\Gamma \frac{\sigma(\tau)}{\tau-\zeta} d\tau
  \end{align}
  \pause
  Other kernels leave us with integrals on the form
  \begin{align}
    \int_\Gamma g(\tau)\log(\tau-\zeta)d\tau \quad \text{ and } \quad
    \int_\Gamma \frac{g(\tau)}{(\tau-\zeta)^q} d\tau, \quad q=1,2,\dots
  \end{align}
  with $g(\tau)$ smooth and implicitly dependent on $\zeta$.

  \pause
  \vg
  With parametrization $\Gamma = \{\gamma(t)\::\:t\in[0, 2\pi)  \}$
  \begin{align}
    \int_\Gamma \frac{g(\tau)}{(\tau-\zeta)^q} d\tau
    =
    \int_0^{2\pi} \frac{g(\gamma(t))}{(\gamma(t)-\zeta)^q} |\gamma'(t)| dt
    =
    \int_0^{2\pi} \frac{f(t)}{(\gamma(t)-\zeta)^q}dt
  \end{align}

  \com{$f(t)=g(\gamma(t))|\gamma'(t)|$ is also assumed smooth}
\end{frame}

\begin{noframe}{Singular quadrature: 2D}
  \begin{itemize}
  \item Smooth quadrature has fixed  $(t_j, w_j)$ for all
    smooth integrands,
    \begin{align}
      \int_a^b f(t) dt = \sum_{j=1}^n f(t_i) w_i
    \end{align}
    \pause
  \item Singular quadrature is tailored on type of quadrature and
    location $t_0$.

    \hg
    E.g. $f(t) = k_0(t) + log|t-t_0| k_L(t)$
    \pause
    \begin{itemize}
    \item \textbf{Product quadrature} integrates the singularity explicitly, $\tilde w_i = \tilde w_i(t_0)$
      \begin{align}
        \int_a^b log|t-t_0| k_L(t) dt = \sum_{j=1}^{n} f_L(t_i) \tilde w_i
      \end{align}
      \pause
    \item \textbf{Implicit rules} integrate the complete expression, $\tilde t_i, \tilde w_i = \tilde t_i(t_0), \tilde w_i(t_0)$      
      \begin{align}
        \text{Modified weights:} &       \int_a^b f(t) dt = \sum_{j=1}^n f(t_i) \tilde w_i \\
        \text{Auxiliary nodes:} &       \int_a^b f(t) dt = \sum_{j=1}^{\tilde{n}} f(\tilde t_i) \tilde w_i \\
      \end{align}
    \end{itemize}

  \end{itemize}
  
\end{noframe}

\begin{frame}{Singular quadrature: 2D Global (PTR)}
  \hfill
  \who{Review by Hao et al. 2013}
\begin{itemize}
\item \textbf{Spectral accuracy:} \\
  \emph{Kress} product quadrature
  $\oint_\Gamma \log\|\xx-\yy\| f(\yy) ds_\yy = \sum_{j=1}^N f(\yy_j) w_j$ \\
  Requires split $k_0(\xx, \yy) + \log\|\xx-\yy\| k_L(\xx, \yy) $
  \vg
  \pause
\item \textbf{$p$th order accuracy:}
  \begin{itemize}
  \item \emph{Kapur–Rokhlin} modifies fixed number of weights near sing.
  \item \emph{Alpert} introduces fixed number of nodes and weights near sing.
    \begin{columns}[c]
      \begin{column}{0.6\textwidth}
        \includegraphics[width=\textwidth]{fig/alpert}
      \end{column}
      \begin{column}{0.3\textwidth}
        \includegraphics[width=\textwidth]{fig/alpert_starfish}
      \end{column}
    \end{columns}
  \end{itemize}
\end{itemize}
\end{frame}

\begin{noframe}{Singular quadrature: 2D Local (G-L panels)}
Maintains $p$th order accuracy of panels.
\hg
\begin{itemize}
\item \emph{Helsing} product quadrature
  \begin{itemize}
  \item For
    $\int_{\tau_a}^{\tau_b} g(\tau)\log(\tau-\zeta)d\tau$
    and
    $\int_{\tau_a}^{\tau_b} \frac{g(\tau)}{(\tau-\zeta)^q} d\tau$
  \item Requires explicit kernel split
  \end{itemize}
  \uncover<2->{
 \item \emph{Generalized Gaussian} quadrature
   \begin{itemize}
   \item Forms nodes and weights through nonlinear optimization
   \item Needs singularity type and location, but not explicit split
   \end{itemize}
   }
\end{itemize}

\vg
\begin{center}
  \includegraphics[height=0.3\textwidth]{fig/panels_nearfar}
    \uncover<2->{
  \includegraphics[height=0.3\textwidth]{fig/gengauss_starfish}
}

  \com{Neighboring panels must be involved.}
\end{center}
\end{noframe}

\begin{frame}{Near singularity?}
  Can understand near singularity through analytic
  continuation of $\gamma(t)$:

  \sg
  \begin{itemize}
  \item   Find preimage $t_0$ such that $\gamma(t_0) = \zeta$ \hfill\com{Done using poly. approx.}
  \begin{align}
    \int_D \frac{f(t) dt }{(\gamma(t)-\zeta)^p}
    = \int_D \frac{f(t) dt }{(\gamma(t)-\gamma(t_0))^p} 
  \end{align}
  \pause
  \item Clear where singularity is \emph{in parametrization} $t$
  \item $t_0$ bounds region of analyticity of integrand
  \end{itemize}
  \begin{center}
    \includegraphics[width=0.4\textwidth]{fig/complexification_gamma_case2.png}
    \hspace{1em} 
    \includegraphics[width=0.4\textwidth]{fig/complexification_t_case2.png}
  \end{center}
\end{frame}

\begin{frame}{Close evaluation errors}
  \begin{center}
    \includegraphics[width=0.4\textwidth]{fig/aqbx_double_lyr_err}
  \begin{columns}[t]
    \begin{column}{0.5\textwidth}
      \begin{center}
        \textbf{Gauss-Legendre panels}\\
        \vspace{1em}
      \includegraphics[width=0.9\textwidth]{fig/aqbx_double_lyr_err_zoom}
    \end{center}
    \end{column}
    \vrule{}
    \begin{column}{0.5\textwidth}
      \begin{center}
        \textbf{Trapezoidal}\\
        \vspace{1em}
      \includegraphics[width=0.9\textwidth]{fig/trapz_complex_zoom}
    \end{center}
    \end{column}
  \end{columns}

\end{center}
\com{Black contours are estimates based on having found $t_0$}
\end{frame}


\begin{frame}{2D: Nearly singular quadrature}
  \begin{itemize}
  \item Many different ideas and methods.
  \item Easiest: Upsampling  to more points
  \item More efficient: Adaptive refinement near target
  \item Very effective in 2D: Helsing quadrature (or variant SSQ based on $t_0$)
  \end{itemize}  
  \vg
  
  \begin{center}
    \begin{columns}[c]
      \begin{column}{0.5\textwidth}
        \only<1>{\includegraphics[width=\textwidth]{fig/compare_2d_direct_k0.25_n16}}
        \only<2>{\includegraphics[width=\textwidth]{fig/compare_2d_direct_k0.25_n32}}        
      \end{column}
      \begin{column}{0.5\textwidth}
        \includegraphics[width=\textwidth]{fig/refinement}        
      \end{column}
    \end{columns}
  \end{center}  
\end{frame}

\begin{noframe}{Interlude: Helsing product quadrature}
  Want to compute nearly singular
  $$
  I = \int_\Gamma \frac{g(\tau)}{(\tau-\zeta)^q} d\tau
  =
  \int_{\tau_a}^{\tau_b} \frac{g(\tau)}{(\tau-\zeta)^q} d\tau
  = \text{\{var. change\}} = 
  \int_{-1}^{1} \frac{g(\tau)}{(\tau-\zeta)^q} d\tau
  $$
  \pause
  \begin{itemize}
  \item Form monomial expansion
    $
    g(\tau) = \sum_{j=0}^{n-1} c_k \tau^k $
    by solving Vandermonde system $V\bm{c} = \bm g$ \hfill \com{Not a problem for $n < 40$}
    \pause
  \item Recursively compute $p_k = \int_{-1}^{1} \frac{\tau^k}{(\tau-\zeta)^q} d\tau$
    \pause
  \item Quadrature is now product
    \begin{align}
      \int_{-1}^{1} \frac{g(\tau)}{(\tau-\zeta)^q} d\tau \approx \sum_{j=0}^{n-1} c_k p_k = \bm{c}^T\bm{p} 
    \end{align}
    \pause
  \item We actually want weights $w_j$ s.t. $I \approx \sum g(t_j) w_j = \bm{g}^T \bm{w}$
  \item These we get by solving adjoint Vandermonde $V^T\bm{w} = \bm p$
  \end{itemize}
  
  
\end{noframe}


\begin{frame}{Interpolation and precomputation}
  \begin{itemize}
  \item Quadrature often ends up needing $\sigma$ at new nodes (upsamling, adaptive refinement, auxiliary nodes, ...)
      \begin{align}
        \int_\Gamma k(\xx, \yy) \sigma(\yy) ds_\yy
        \approx
        \sum_{j=1}^{\tilde n} k\left(\xx, \yy(\tilde t_j)\right) \sigma(\tilde t_j) |J_\yy(\tilde t_j)| \tilde w_j = \bm{\tilde a}^T\bm{\tilde\sigma}, 
      \end{align}
      \pause
    \item Density at new nodes we get through interpolation
      $$\bm{\tilde\sigma} = \underbrace{P}_{\tilde n \times n}\bm{\sigma}, \quad \tilde n > n$$
      \pause
    \item If reusable, we can compress this into $\bm{a} = P^T\bm{\tilde a}$ such that
      \begin{align}
        \bm{a}^T\bm{\sigma} = \bm{\tilde a}^T\bm{\tilde\sigma}
      \end{align}
    \item $\bm{a}$ corresponds to modified row entries in $A$ matrix
  \end{itemize}
\end{frame}



\begin{noframe}{QBX}
  Very different idea: Quadrature By eXpansion. For 2D Laplace DLP
  \begin{align}
    u(\zeta) = 
    \int_\Gamma \frac{g(\tau)}{(\tau-\zeta)} d\tau =
    \sum_{m=0}^\infty (\zeta-\zeta_0)^m
    \underbrace{
    \int_\Gamma \frac{g(\tau)}{(\tau-\zeta_0)^{m+1}} d\tau
    }_{c_m(\zeta_0)}
  \end{align}
  for $|\zeta-\zeta_0| \le \sup_{\tau\in\Gamma} |\tau-\zeta_0|$.
  \pause
  Same idea extends to 3D.
  \begin{columns}
    \begin{column}{0.5\textwidth}
      \only<1>{
        \includegraphics[width=\textwidth]{fig/spheroid_neval}
      }
      \only<2>{
        \includegraphics[width=\textwidth]{fig/spheroid_neval_punc}
      }
      \only<3->{
        \includegraphics[width=\textwidth]{fig/spheroid_neval_qbx}
      }

    \end{column}
    \begin{column}{0.49\textwidth}
      \only<4>{
      \includegraphics[width=\textwidth]{fig/spheroid_grid_expansion_point}
    }
    \only<5>{
      \includegraphics[width=\textwidth]{fig/upsampled_grid_expansion_point}
    }
    \end{column}
  \end{columns}
\end{noframe}

\begin{frame}{Hedgehog}
  \begin{itemize}
  \item Same underlying idea as QBX: Expand solution in domain
  \item Here: Along a line
  \item Simple, but hard to optimize
  \end{itemize}
  \begin{center}
    \includegraphics[width=0.5\textwidth]{fig/hedgehog}
  \end{center}
\end{frame}


\begin{noframe}{3D: Everything harder}
  Weak sing. integrable if at center of polar coord. system on surface
  \begin{align}
    \int_D k(\xx, \yy(r, \theta)) \sigma(r, \theta) |J_\yy(r, \theta)| r dr d\theta
  \end{align}

  \pause
  \vg
  \textbf{Global discretization:}

  Grid rotations / Local patch / High-order correction
  \begin{center}
    \includegraphics[height=0.3\textwidth]{fig/grid_orig}
    \includegraphics[height=0.3\textwidth]{fig/grid_patch}
    \includegraphics[height=0.3\textwidth]{fig/pou}
    \includegraphics[height=0.3\textwidth]{fig/torus_local_corr}
  \end{center}

  \vg
  Other approaches: Regularization, singularity subtraction (limited order)
\end{noframe}

\begin{frame}
  \textbf{Local discretization:} \com{As used in FMM3DBIE}

  Singular: Auxiliary nodes from generalized Gaussian quadrature
  \begin{center}
    \includegraphics[height=0.2\textwidth]{fig/bremer_triangle}
  \end{center}
  \pause
  Close eval: Adaptive refinement
  \begin{center}
    \includegraphics[height=0.2\textwidth]{fig/panel_adap_ref}
  \end{center}  
\end{frame}

\begin{frame}
  New alternative: Product quadrature using Stokes theorem
  \begin{center}
    \includegraphics[width=0.8\textwidth]{fig/hai_product_quad}
  \end{center}
\end{frame}

\begin{frame}{Volume potentials}
  Did not mention these much.
  \begin{align}
    V[\sigma](\xx) = \int_\Omega k(\xx, \yy) \sigma(\yy) ds_\yy
  \end{align}
  \begin{center}
    \includegraphics[width=0.8\textwidth]{fig/anderson_volpot}
  \end{center}
  \hfill\who{Anderson et al.}
\end{frame}

\begin{frame}{Looking forward}
  \begin{itemize}
  \item 2D can be considered solved, focus on 3D!
  \item Wishlist:
    \begin{itemize}
    \item Close eval \& On-surface \& Volume
    \item High order
    \item Fast, parallelizable (moving / deforming geometries)
    \item Robust w.r.t. geometry
    \item Generalizable to the full zoo of kernels
    \end{itemize}
  \end{itemize}
  \vfill
  \who{Apologies for figures stolen and credit not given.}
\end{frame}

\begin{frame}[c]
  \begin{align}
    \square
  \end{align}
\end{frame}

\end{document}
